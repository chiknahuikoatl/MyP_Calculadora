\documentclass[a4paper,12pt]{report}

\usepackage[spanish]{babel}
\usepackage[utf8]{inputenc}
\usepackage{amsmath}
\usepackage{fancyhdr}
\usepackage{graphicx}
\graphicspath{ {imagenes/} }

\usepackage{hyperref}
\hypersetup{
    colorlinks=true,
    linkcolor=blue,
    filecolor=magenta,
    urlcolor=cyan,
}

\title{\bf Tarea 6}
\author{Jerónimo Almeida Rodríguez}
\author{Martin Felipe Espinal Cruces}
\date{\today}

\pagestyle{fancy}
\lhead{Almeida \& Espinal}
\chead{Tarea 6}
\rhead{\today}
\lfoot{jalrod@ciencias.unam.mx}
\rfoot{cofy43@ciencias.unam.mx}

\begin{document}
\begin{titlepage}
    \centering
    {\scshape\Huge Universidad Nacional Autónoma de México \par}
    \vspace{2cm}
    {\scshape\huge Modelado y Programación\par}
    \vspace{2cm}
    {\huge\bfseries Tarea 6\par}
    \vspace{1.5cm}
    {\Large\textsc Jerónimo Almeida Rodríguez \par}
    \vspace{.25cm}
    {\large\texttt{ jalrod@ciencias.unam.mx}\par}
    \vspace{1cm}
    {\Large\textsc Martin Felipe Espinal Cruces \par}
    \vspace{.25cm}
    {\large\texttt{cofy43b@ciencias.unam.mx}\par}
    \vspace{2cm}
    \vfill
    \begin{figure}[hb!]
        \includegraphics[width=.3\textwidth]
            {../../logos/escudo_f-ciencias.png}\hfill
        \includegraphics[width=.3\textwidth]
            {../../logos/Escudo_UNAM.png}\hfill
    \end{figure}
\end{titlepage}

\section*{Planteamiento del Problema.}{
    El planteamiento del problema se encuentra en la especificación de la tarea.
}
\section*{Objetivo.}{
    Desarrollar una interfaz gráfica e implementar operaciones para una
    calcualdora con el objetivo de entender el proceso de un compilador. Esto
    mediante el procesamiento de la expresión aritmética dada.
}
\section*{Desarrollo.}{
    Lo primero que se hizo fue hacer la interfaz gráfica con los botones
    correspondientes a cada elemento a implementar.\\
    Después, se creó el archivo \texttt{ControladorVista.java} dónde cada
    símbolo se introduciría a la cadena que representa la expresión aritmética.
    Al precionar el botón de $=$, la cadena se envía al modelo para ser
    procesada e imprime en la pantalla el resultado de la evaluación en caso de
    que la expresión sea correcta. En otro caso imprime un error.\\
    Para evaluar los operadores extras se añadieron al \texttt{StringTokenizer}
    de la clase \texttt{NodoOperador.java} representaciones en cadena de dichas
    operaciones. Después se crearon sus nodos correspondientes tomando en cuenta
    su precedencia y se añadieron a la fábrica de métodos del compilador.\\
    Finalmente se añadió su representación en cadena en la clase
    \texttt{NodoOperador.java}. Cabe mencionar que la representacion en cadena
    de un opjeto de tipo \texttt{CompositeEA} es distinta a la de la espresión
    aritmética por fines de practicidad, para poder procesar de manera más
    sencilla los operadores.\\
    Nota: Al intentar generar la documentación desde NetBeans surgieron muchos
    errores por el formato de las clases del compilador y CompositeEA.
}
\section*{Solución.}{
    Además de lo anteriormente mencionado y de lo especificado en la
    especificación, se añadieron los siguientes operadores:
    \begin{align*}
        < &:= \text{mueve el cursor un caracter a la izquierda.}\\
        > &:= \text{mueve el cursor un caracter a la derecha.}\\
        \pi &:= \text{procesa la constante pi}\\
        *Backspace &:= \text{Borra el caracter anterior (a la izquierda).}
    \end{align*}
}

\begin{thebibliography}{}
\end{thebibliography}
\end{document}
